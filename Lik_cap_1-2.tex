\documentclass[10pt]{beamer}\usepackage{graphicx, color}
%% maxwidth is the original width if it is less than linewidth
%% otherwise use linewidth (to make sure the graphics do not exceed the margin)
\makeatletter
\def\maxwidth{ %
  \ifdim\Gin@nat@width>\linewidth
    \linewidth
  \else
    \Gin@nat@width
  \fi
}
\makeatother

\definecolor{fgcolor}{rgb}{0.2, 0.2, 0.2}
\newcommand{\hlfunctioncall}[1]{\textcolor[rgb]{0,0,0.545098039215686}{\textbf{#1}}}%
\newcommand{\hlstring}[1]{\textcolor[rgb]{0.282352941176471,0.23921568627451,0.545098039215686}{#1}}%
\newcommand{\hlnumber}[1]{\textcolor[rgb]{0,0,0}{#1}}%
\newcommand{\hlkeyword}[1]{\textcolor[rgb]{0,0,0}{\textbf{#1}}}%
\newcommand{\hlargument}[1]{\textcolor[rgb]{0.690196078431373,0.250980392156863,0.0196078431372549}{#1}}%
\newcommand{\hlcomment}[1]{\textcolor[rgb]{0.2,0.2,0.2}{#1}}%
\newcommand{\hlroxygencomment}[1]{\textcolor[rgb]{0.2,0.2,0.2}{#1}}%
\newcommand{\hlformalargs}[1]{\textcolor[rgb]{0.690196078431373,0.250980392156863,0.0196078431372549}{#1}}%
\newcommand{\hleqformalargs}[1]{\textcolor[rgb]{0.690196078431373,0.250980392156863,0.0196078431372549}{#1}}%
\newcommand{\hlassignement}[1]{\textcolor[rgb]{0,0,0}{\textbf{#1}}}%
\newcommand{\hlpackage}[1]{\textcolor[rgb]{0.588235294117647,0.709803921568627,0.145098039215686}{#1}}%
\newcommand{\hlslot}[1]{\textit{#1}}%
\newcommand{\hlsymbol}[1]{\textcolor[rgb]{0,0,0}{#1}}%
\newcommand{\hlprompt}[1]{\textcolor[rgb]{0.2,0.2,0.2}{#1}}%

\usepackage{framed}
\makeatletter
\newenvironment{kframe}{%
 \def\at@end@of@kframe{}%
 \ifinner\ifhmode%
  \def\at@end@of@kframe{\end{minipage}}%
  \begin{minipage}{\columnwidth}%
 \fi\fi%
 \def\FrameCommand##1{\hskip\@totalleftmargin \hskip-\fboxsep
 \colorbox{shadecolor}{##1}\hskip-\fboxsep
     % There is no \\@totalrightmargin, so:
     \hskip-\linewidth \hskip-\@totalleftmargin \hskip\columnwidth}%
 \MakeFramed {\advance\hsize-\width
   \@totalleftmargin\z@ \linewidth\hsize
   \@setminipage}}%
 {\par\unskip\endMakeFramed%
 \at@end@of@kframe}
\makeatother

\definecolor{shadecolor}{rgb}{.97, .97, .97}
\definecolor{messagecolor}{rgb}{0, 0, 0}
\definecolor{warningcolor}{rgb}{1, 0, 1}
\definecolor{errorcolor}{rgb}{1, 0, 0}
\newenvironment{knitrout}{}{} % an empty environment to be redefined in TeX

\usepackage{alltt}

\usetheme[compress]{PaloAlto}
\usecolortheme{sidebartab}
%\logo{\includegraphics[width=1cm]{../Rlogo-5.png}}

\usepackage[brazilian]{babel}
\usepackage[T1]{fontenc}
\usepackage[utf8]{inputenc}
\usepackage{graphicx}
\usepackage{hyperref}
\usepackage[scaled]{beramono} % truetype: Bistream Vera Sans Mono
%\usepackage{inconsolata}

\setbeamertemplate{footline}[frame number] % mostra o numero dos slides
\setbeamertemplate{navigation symbols}{} % retira a barra de navegacao

\usepackage{xspace}
\providecommand{\eg}{\textit{e.g.}\xspace}
\providecommand{\ie}{\textit{i.e.}\xspace}
\providecommand{\R}{\textsf{R}\xspace}


\title[Verossimilhança]{Verossimilhança --- Capítulos 1 e 2\\
  Revisão de resultados da Teoria de Probabilidades}
\author[MATH 232]{}
%% \institute[Universidade Federal de Santa Catarina (UFSC)]{Departamento
%%   de Ecologia e Zoologia (ECZ) \\ Universidade Federal de Santa Catarina
%%   (UFSC)}
\date{\today}
%\logo{\includegraphics[width=1cm]{../vertical_sigla_fundo_claro.pdf}}

\AtBeginSection[]
{
  \begin{frame}
    \frametitle{Sumário}
    \tableofcontents[currentsection]
  \end{frame}
}

\AtBeginSubsection[]
{
  \begin{frame}
    \frametitle{Sumário}
    \tableofcontents[currentsection,currentsubsection]
  \end{frame}
}
\IfFileExists{upquote.sty}{\usepackage{upquote}}{}

\begin{document}




\begin{frame}
\maketitle
%\titlepage
\end{frame}

\begin{frame}{Sumário}
\tableofcontents
\end{frame}

\section{Variáveis aleatórias}
\label{sec:va}

\begin{frame}{Variáveis aleatórias}
  Vamos considerar apenas variáveis quantitativas (qualitativas devem
  ser tratadas de maneira apropriada \ldots)\\
  \textbf{Definição ``intuitiva''}
  \begin{itemize}
  \item Uma \textbf{variável} é uma característica de uma população que
    pode ser mensurada
  \item Uma \textbf{variável} é dita \textbf{aleatória} quando não
    podemos \underline{determinar} seu resultado (embora possamos
    \underline{prever}, conforme será mostrado adiante)
  \end{itemize}
\end{frame}

\begin{frame}{Variáveis aleatórias}
  \textbf{Definição ``fraca''} (Meyer, Dantas)
  \begin{itemize}
  \item Variável Aleatória (VA) é uma função $X(\omega)$, de um evento
    (realização) $\omega$, definida em um espaço
    amostral $\Omega$, que assume valores reais $\mathbb{R}$
    \begin{equation*}
      X(\omega) = \omega;\ \qquad X(\omega) \in \mathbb{R},\ \omega
      \in \Omega
    \end{equation*}
  \item Se a VA $X(\omega)$ assume valores enumeráveis (não
    necessariamente finitos), são denominadas \textbf{discretas}
  \item Se a VA $X(\omega)$ assume valores em um intervalo da reta real,
    são denominadas \textbf{contínuas}
  \end{itemize}
\end{frame}

\begin{frame}{Variáveis aleatórias}
  \textbf{Definição ``forte'' (formal)} (Magalhães, James)
  \begin{itemize}
  \item Definida em função de $\sigma$-álgebra, \ldots
  \item[-] Não
  \end{itemize}
\end{frame}

\subsection[Discretas]{Variáveis aleatórias discretas}
\label{sec:vadisc}

\begin{frame}{Variáveis aleatórias discretas}
  Uma VA \textbf{discreta} é uma função que atribui uma probabilidade a
  cada valor específico. É denominada \textit{função discreta de
    probabilidade}, \textit{função massa de probabilidade} e é denotada por
  \begin{equation*}
    P(X = x_i) = p(x_i) = p_i, \quad i = 1, 2, \ldots
  \end{equation*}
  e tem as seguintes propriedades:
  \begin{enumerate}
  \item $\sum_{i} P(X = x_i) = 1$
  \item $0 \leq P(X = x_i) \leq 1$
  \end{enumerate}
\end{frame}

\begin{frame}{Variáveis aleatórias discretas}
  \begin{itemize}
  \item Exemplos \ldots
  \item Principais ``funções'' discretas
    \begin{itemize}
    \item Binomial
    \item Poisson
    \item Hipergeométrica
    \item Binomial negativa
    \end{itemize}
  \end{itemize}
\end{frame}

\subsection[Contínuas]{Variáveis aleatórias contínuas}
\label{sec:vacont}

\begin{frame}{Variáveis aleatórias contínuas}
  \begin{itemize}
  \item Para uma VA \textbf{contínua}, não é possível atribuir uma
    probabilidade a um valor específico, pois existe uma quantidade não
    enumerável de valores positivos em um ponto da reta, e portanto a
    soma não poderia ser 1.
  \item Por isso, devemos atribuir probabilidades à \textbf{intervalos} de
    valores da VA contínua, por meio da função $f(x)$ que a
    descreve. Essa função deve ter as seguintes propriedades:
    \begin{enumerate}
    \item $f(x) \geq 0 \quad \forall\, x \in (-\infty,\infty)$
    \item $\int_{-\infty}^{\infty} f(x) dx = 1$
    \end{enumerate}
    Portanto, para calcular uma probabilidade para um intervalo entre $a
    \leq b$, temos a relação
    \begin{equation*}
      P(a \leq X \leq b) = \int_{a}^{b} f(x) dx
    \end{equation*}
  \end{itemize}
\end{frame}

\begin{frame}{Variáveis aleatórias contínuas}
  VAs contínuas também podem ser definidas pela \textit{função de
    distribuição acumulada} $F(x)$,
  \begin{equation*}
    F(X) = P(X \leq x)
  \end{equation*}
  que possui as seguintes propriedades
  \begin{enumerate}
  \item $0 \leq F(x) \leq 1$
  \item $F(x)$ é não decrescente, e contínua à direita
  \item $\lim_{x \rightarrow -\infty} F(x) = 0$ e
    $\lim_{x \rightarrow \infty} F(x) = 1$
  \end{enumerate}
\end{frame}

\begin{frame}{Variáveis aleatórias contínuas}
  Pelas definições da fda, podemos agora calcular probabilidades entre
  $x$ e $x + dx$ (onde $dx > 0$ é um valor incremental), através de
  \begin{equation*}
    P(x \leq X \leq x + dx) = F(x + dx) - F(x)
  \end{equation*}
\end{frame}

\begin{frame}{Variáveis aleatórias contínuas}
  \begin{itemize}
  \item Exemplos \ldots
  \item Principais ``funções'' contínuas
    \begin{itemize}
    \item Beta
    \item Exponencial
    \item Gama
    \item Uniforme
    \item Normal
    \item t-Student
    \end{itemize}
  \end{itemize}
\end{frame}


\section{Variáveis aleatórias bidimensionais}
\label{sec:vabi}

\begin{frame}{Variáveis aleatórias bidimensionais}

\end{frame}

\section{Esperança de variáveis aleatórias}
\label{sec:eva}

\begin{frame}{Esperança de variáveis aleatórias}

\end{frame}

\section{Distribuições de probabilidade}

\begin{frame}[fragile=singleslide]{Distribuições de probabilidade}
A maioria das distribuições de probabilidade tradicionais estão
implementadas no R, e podem ser utilizadas para substituir as tabelas
estatísticas tradicionais. Existem 4 itens fundamentais que podem ser
calculados para cada distribuição:
\begin{itemize}
\item[d*] Calcula a densidade de probabilidade ou probabilidade pontual
\item[p*] Calcula a função de probabilidade acumulada
\item[q*] Calcula o quantil correspondente a uma dada probabilidade
\item[r*] Gera números aleatórios (ou ``pseudo-aleatórios'')
\end{itemize}
\end{frame}

\begin{frame}[fragile=singleslide]{Distribuições de probabilidade}
As distribuições de probabilidade mais comuns são:
\begin{center}
% use packages: array
\begin{tabular}{lll}
\hline
Distribuição & Nome no \R & Parâmetros \\
\hline
Binomial & \texttt{*binom} & \texttt{size, prob} \\
$\chi^2$ & \texttt{*chisq} & \texttt{df} \\

Normal & \texttt{*norm} & \texttt{mean, sd} \\
Poisson & \texttt{*pois} & \texttt{lambda} \\
t & \texttt{*t} & \texttt{df} \\
Uniforme & \texttt{*unif} & \texttt{min, max}\\
\hline
\end{tabular}
\end{center}
\end{frame}

\begin{frame}[fragile=singleslide]{Distribuições de probabilidade}
Aplicando no R
\begin{knitrout}\small
\definecolor{shadecolor}{rgb}{0.933, 0.914, 0.914}\color{fgcolor}\begin{kframe}
\begin{alltt}
> \hlfunctioncall{rnorm}(10)
\end{alltt}
\begin{verbatim}
 [1]  0.048592  0.219971 -0.402351  0.147845  1.167548
 [6] -0.924155 -0.173920  0.719823  1.522894 -0.595841
\end{verbatim}
\begin{alltt}
> \hlfunctioncall{hist}(\hlfunctioncall{rnorm}(100))
\end{alltt}
\end{kframe}

{\centering \includegraphics[width=.4\textwidth]{figure/unnamed-chunk-1} 

}



\end{knitrout}

\end{frame}

\begin{frame}[fragile=singleslide]{Distribuições de probabilidade}
\begin{knitrout}\small
\definecolor{shadecolor}{rgb}{0.933, 0.914, 0.914}\color{fgcolor}

{\centering \includegraphics[width=.7\textwidth]{figure/unnamed-chunk-2} 

}



\end{knitrout}

\end{frame}

\begin{frame}[fragile=singleslide]{Distribuições de probabilidade}
\begin{knitrout}\small
\definecolor{shadecolor}{rgb}{0.933, 0.914, 0.914}\color{fgcolor}

{\centering \includegraphics[width=.49\textwidth]{figure/unnamed-chunk-31} 
\includegraphics[width=.49\textwidth]{figure/unnamed-chunk-32} 

}



\end{knitrout}

\end{frame}

\begin{frame}[fragile=singleslide]{Distribuições de probabilidade}
Alguns exemplos:
\begin{knitrout}\small
\definecolor{shadecolor}{rgb}{0.933, 0.914, 0.914}\color{fgcolor}\begin{kframe}
\begin{alltt}
> \hlcomment{# quantis de z}
> \hlfunctioncall{qnorm}(0.025)
\end{alltt}
\begin{verbatim}
[1] -1.96
\end{verbatim}
\begin{alltt}
> \hlfunctioncall{qnorm}(0.975)
\end{alltt}
\begin{verbatim}
[1] 1.96
\end{verbatim}
\begin{alltt}
> \hlcomment{# quantis de t com diferentes G.L.}
> \hlfunctioncall{qt}(0.025, df = 9)
\end{alltt}
\begin{verbatim}
[1] -2.2622
\end{verbatim}
\begin{alltt}
> \hlfunctioncall{qt}(0.025,df = 900)
\end{alltt}
\begin{verbatim}
[1] -1.9626
\end{verbatim}
\end{kframe}
\end{knitrout}

\end{frame}


\section{Teorema do Limite Central}
\label{sec:tlc}

\begin{frame}{Teorema do Limite Central}

\end{frame}


\end{document}
